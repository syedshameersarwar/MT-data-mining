The next phase of analysis after evaluating hypothesis and determing drug significance 
is to determine the relative effect of the drug on the biological feautres and 
how they are varying under difference concentrations of the drug. This is done by the application
of Hill fitting to model the drug response curves and finding the EC50/IC50 values (Section \ref{sec:hillfitting}). The EC50 values
are the concentration of the drug at which the response is half of the maximum response. The IC50 values
are the concentration of the drug at which the response is half of the maximum inhibition.
The determined EC50/IC50 values are used for relative comparison (Section \ref{sec:relative_comparison}) of the drug effect on the biological features to baseline.
Finally, a systematic strategy for finding out features which are behaving similar to different drug concentrations based on 
Principal Components Influence Analysis is presented in (Section \ref{sec:pcia}).


\section {Hill Fitting}
\label{sec:hillfitting}
Hill fitting is a method used to model the drug response curves. The Hill equation is given by:
## convert it into latex equation
E/E_max = [A]^n / (EC^50 + [A]^n)
where E is the response, E_max is the maximum response, A is the concentration of the drug, EC50 is the concentration of the drug at which the response is half of the maximum response and n is the Hill coefficient. The Hill coefficient is a measure of the steepness of the curve. 

For our analysis, we selected four features, namely Field potential Duration (shown as duration), Force and Calcium Peak Amplitudes and Local Frequency and fitted the 
Hill equation to Calcium Titration experiement and Nifedipine data. E-4031 was not considered as it only had three concentrations and was not enough to fit the Hill equation. 
The response were average over different tissues and normalized between 0 and 1. The Hill equation was fitted using the Hillfit (\ref{link:hillfit}) package in Python. (Figure \ref{fig:hill_analysis}).
The EC50/IC50 values were determined from the Hill fitting and the closest available concentration value to the mean of these values was taken as the final EC50/IC50 value for the drug. It was 1.0 um For both Nifedipine (~0.91 um) and Calicum Titration (1.22 um). The EC50/IC50 values were then used for relative comparison of the drug effect on the biological features to baseline in the next section.

\section {Relative Comparison of Drug Effect}
\label{sec:relative_comparison}
The EC50/IC50 values were used for relative comparison of the drug effect on the biological features to baseline. The baseline experiments setup for each drug was merge 
and the mean of the feature values was calculated and kept at 100%. The mean of the feature values across tissue at the determined EC50/IC50 concentration from previous section was determined.
The percent relative change of these values from the baseline was calculated and plotted in Figure \ref{fig:mean_relative_change_baseline_ec50}. For E-4031, the largest concentration available (0.003 um) was used instead of the EC50 value. 

This analysis aid in characterizing the drug effect on the biological features in comparision to baseline. 

\section {Principal Components Influence Analysis}
\label{sec:pcia}
After determining relative drug effects, a systematic approch to find out features which are behaving similar to different drug concentrations i.e they are varying in a similar manner is presented.

The methodology is based on application of Prinicipal Component Analysis (PCA) and DBSCAN. Specifically, the PCA is applied to the data for each drug containing different concentrations.
The components with 0.95 variance are selected and the feature contribution (i.e Loadings) to these components are clustered using DBSCAN. 
The parameters of DBSCAN are set to min_samples=2 as even if two features are behaving similar, they should be considered. 
For determining the epsilon (eps) value, the k-nearest neighbors distance for the feautres are calculated with K set to 2 (\ref{cite:dbscan-heruistics}). 
The knee point of sorted distance plot was determined through Kneedle algorithm [\ref{link:kneedle}] and the epsilon value was set to the distance at the knee point.
